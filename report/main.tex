\documentclass{article}
\usepackage{listings}
\usepackage{booktabs}
\usepackage{here}
\usepackage{subcaption}
\usepackage{tikz}
\usepackage{amsmath}
\usepackage{graphicx}
\usepackage{hyperref}
\usepackage{siunitx} % Required for the \num command


\newcommand{\code}[1]{{\tt #1}}

% decreas margin size
\usepackage[margin=0.7in]{geometry}

\newcommand\blankpage{%
	\null
	\thispagestyle{empty}%
	\addtocounter{page}{-1}%
	\newpage}

\title{
	\normalfont\normalsize
	\textsc{Data Analytics (2024-25)\\
	Universit\`a della Svizzera italiana}\\
	% \vspace{2pt}
	\rule{\linewidth}{0.5pt}\\
	% \vspace{5pt}
	{\huge Dating Profile\\
	\small Course Assignment N.16}\\
	% \vspace{5pt}
	\rule{\linewidth}{1pt}\\
	\vspace{5pt}
}

\author{
	Paolo Deidda (\text{paolo.deidda@usi.ch}) \\ 
	Pareek Yash (\text{yash.pareek@usi.ch})\\
	\url{https://github.com/USI-Projects-Collection/DA-Dating-Profile.git}
	}


\begin{document}
\maketitle

\tableofcontents

% \vspace{50pt}
% \rule{\linewidth}{0.5pt}

\vspace*{\fill}

\subsection*{Setup}

To run the code and reproduce the figures or outputs, you can either run the Jupyter Notebook directly on Google Colab, or follow the setup and execution instructions provided in the \texttt{README.md} file included in this repository.

\section*{Introduction}

This report documents our approach to analyzing and building recommendation systems based on a dataset of user ratings for dating profiles. The objective of this assignment is to explore and preprocess the data, then develop and evaluate different recommender system models capable of predicting user preferences.

The dataset comprises over 3.2 million ratings linking users to dating profiles, accompanied by auxiliary gender data. We begin with Exploratory Data Analysis (EDA) to understand the distribution and structure of the data, followed by data preprocessing to clean and optimize it for modeling. Subsequently, we construct three recommendation approaches: a naive baseline, a collaborative filtering model using item-item k-nearest neighbors, and a content-based filtering model leveraging profile attributes. Performance is assessed using the Mean Absolute Error (MAE) metric, and results are compared to determine the most effective strategy for this domain.


\newpage

\section{Data Exploration - EDA}
% This file contains the data exploration section of the report.
\subsubsection*{DataFrame Overview}
\begin{table}[ht]
\centering
\caption{Structure of the Ratings DataFrame}
\label{tab:df-overview}
\begin{tabular}{@{}ll@{}}
\toprule
Property        & Value                                             \\ 
\midrule
\# Rows         & 3{,}220{,}037                                     \\
\# Columns      & 3 (\texttt{userID}, \texttt{profileID}, \texttt{rating})    \\
Data types      & \texttt{int64}, \texttt{int64}, \texttt{int64}   \\
Memory usage    & 73.7\,MB                                          \\
\bottomrule
\end{tabular}
\end{table}

\subsubsection*{Missing Values and Duplicates}
\begin{table}[ht]
\centering
\caption{Counts of Missing and Duplicate Rows}
\label{tab:missing-dup}
\begin{tabular}{@{}lrr@{}}
\toprule
Column        & Missing values & Duplicate rows \\ 
\midrule
\texttt{userID}    & 0              & 47   \\
\texttt{profileID} & 0              & 0    \\
\texttt{rating}    & 0              & 0    \\
\bottomrule
\end{tabular}
\end{table}

\noindent\textbf{Unique entities.} After deduplication, there are 25\,245 unique users and 125\,428 unique profiles.

\subsubsection*{Rating Distribution}

\begin{table}[ht]
    \centering
    \caption{Descriptive Statistics of the \texttt{rating} Column}
    \label{tab:rating-stats}
    \begin{tabular}{@{}lrrrrrrrr@{}}
    \toprule
    Statistic   & Count        & Mean   & Std    & Min & 25\% & 50\% & 75\% & Max \\ 
    \midrule
    Rating      & 3{,}220{,}037 & 5.9532 & 3.1064 & 1  & 3    & 6  & 9    & 10  \\
    \bottomrule
    \end{tabular}
\end{table}

Figure~\ref{fig:hist-rating} shows the histogram of all ratings, confirming a mild skew towards higher scores.

\begin{figure}[ht!]
  \centering
  \includegraphics[width=0.6\linewidth]{figures/hist.png}
  \caption{Histogram of Profile Ratings}
  \label{fig:hist-rating}
\end{figure}

\subsubsection*{User Activity}
\begin{table}[ht]
    \centering
    \caption{Number of Ratings Given per User}
    \label{tab:user-stats}
    \begin{tabular}{@{}lrrrrrrrr@{}}
    \toprule
    Statistic        & Count    & Mean    & Std     & Min & 25\% & 50\% & 75\%   & Max     \\ 
    \midrule
    Ratings per user & 25{,}245 & 127.55  & 362.10  & 2   & 29   & 73   & 123    & 18{,}342 \\
    \bottomrule
    \end{tabular}
\end{table}


\subsubsection*{Outlier Detection}
Using the IQR method ($Q_1 - 1.5\cdot\mathrm{IQR}$, $Q_3 + 1.5\cdot\mathrm{IQR}$), \emph{no} ratings fell outside the acceptable range, indicating the absence of extreme outliers.

\subsubsection*{Correlation Analysis}
We computed the Pearson correlation between the total number of ratings a profile received and its average rating:
\[
  \mathrm{Corr}(\text{rating\_count},\,\text{avg\_rating})
  \;=\;0.0201,
\]
a very weak positive relationship. The corresponding scatter plot (Figure~\ref{fig:rating-vs-count}) shows no strong trend.

\begin{figure}[ht]
  \centering
  \includegraphics[width=0.6\linewidth]{figures/output.png}
  \caption{Average Rating vs.\ Number of Ratings per Profile}
  \label{fig:rating-vs-count}
\end{figure}

\section{Data Preprocessing}
\subsection*{Raw files and initial footprint}

The original dataset consists of three files:

\begin{itemize}
  \item \texttt{ratings.dat} – \num{3220037} user--item interactions
        (\textit{userID}, \textit{profileID}, \textit{rating})\@.
  \item \texttt{ratings\_Test.dat} – the held-out test split
        (\num{276053} rows, same schema).
  \item \texttt{gender.dat} – \num{220970} user--gender pairs.
\end{itemize}

Loaded with Pandas’ default \texttt{int64} dtypes the two training files occupy
$\sim\!90$\,MB of memory.

\subsection*{Dtype optimisation}

\vspace{2pt}
\begin{tabular}{@{}lcc@{}}
\toprule
\textbf{Column} & \textbf{Original dtype} & \textbf{Final dtype} \\ \midrule
\textit{userID}, \textit{profileID} & \texttt{int64} & \texttt{int64}\textsuperscript{*} \\
\textit{rating}                     & \texttt{int64} & \texttt{float32} \\
\textit{gender}                     & \texttt{int64} & \texttt{category} \\ \bottomrule
\end{tabular}

\smallskip
\noindent\textsuperscript{*}\,Required by \texttt{torch.nn.Embedding}.
Casting the other two columns shrinks the in-RAM size of
\texttt{ratings.dat} to $\sim\!61$\,MB and \texttt{gender.dat} to
$\sim\!2$\,MB (a 74\,\% reduction overall).

\subsection*{Duplicate removal}

A scan for exact duplicates uncovered \num{47} repeated
\textit{(user,\,profile)} pairs in the training split; these rows were
dropped, leaving \num{3219990} unique ratings.  
No missing values were present in any file.

\subsection*{Persisting the processed data}

The cleaned frames are serialised to \texttt{.pkl} with
\texttt{DataFrame.to\_pickle()}, bypassing expensive CSV parsing in every
notebook run.


% \newpage

\section{Recommender Systems}
\subsection{Naive Model}
\subsection*{Model definition}

Two simple, parameter-free baselines are computed:

\begin{enumerate}
  \item \textbf{Global mean}  
        \(\hat r_{ui} = \mu\), the average of \emph{all} ratings.
  \item \textbf{Item mean}  
        \(\hat r_{ui} = \bar r_{i}\); if an item is unseen, fall back to
        the global mean.
\end{enumerate}

\subsection*{Results}

\begin{table}[h]
  \centering
  \begin{tabular}{@{}lcc@{}}
    \toprule
    \textbf{Baseline} & \textbf{Evaluation split} & \textbf{MAE} \\ \midrule
    Global mean & test & 2.6545 \\
    Item mean   & test     & 1.4620 \\ \bottomrule
  \end{tabular}
  \caption{Performance of naïve predictors.}
  \label{tab:naive}
\end{table}

The item-mean strategy reduces the error by roughly 45\,\% relative to the
global average, establishing a strong but effort-free reference.

\subsection{Collaborative Filtering}
\subsection*{Model formulation}

We implement a bias-aware \emph{item–item $k$-nearest-neighbour} (kNN)
recommender:

\[
  \mu = \frac{1}{|R|} \sum_{(u,i)\in R} r_{ui}, \qquad
  b_u = \bar r_u - \mu, \qquad
  b_i = \bar r_i - \mu .
\]

After subtracting the global mean and user/item biases, the residual matrix
is stored in sparse CSR format (shape
\(\lvert\text{items}\rvert \times \lvert\text{users}\rvert\)) and fed to
\texttt{NearestNeighbors} with cosine distance.

For a target pair \((u,i)\) the prediction rule is

\[
  \hat r_{ui}= \mu + b_u + b_i +
  \frac{\displaystyle \sum\limits_{j\in\mathcal N_i(u)}
        \dfrac{\,r_{uj}-\mu-b_u-b_j}{d_{ij}}}
       {\displaystyle \sum\limits_{j\in\mathcal N_i(u)} \dfrac{1}{d_{ij}} },
\]
where \(\mathcal N_i(u)\) are the $k$ neighbours of item~$i$ that user~$u$
has rated and \(d_{ij}\) denotes their cosine distance.

\subsection*{Hyper-parameter selection}

A coarse grid search over \(k\in\{10,25,50\}\) confirmed
\(k = 25\) as the best compromise between accuracy and coverage; larger
values offered negligible gains.

\subsection*{Evaluation}

\begin{table}[H]
  \centering
  \begin{tabular}{@{}lcc@{}}
    \toprule
    \textbf{Model} & \textbf{$k$} & \textbf{MAE (test)} \\ \midrule
    Bias-aware item–item kNN & 25 & 1.4633 \\ \midrule
    Item mean baseline (Table~\ref{tab:naive}) & -- & 1.4620 \\ \bottomrule
  \end{tabular}
  \caption{Collaborative filter vs.\ strongest naïve baseline.}
  \label{tab:cf}
\end{table}

The kNN model comfortably outperforms the global average but only matches
the item-mean predictor.  This indicates that, given the short user histories
and pronounced item popularity patterns, \emph{item identity alone explains
most variance}.  Further improvement is likely to require latent-factor
methods or hybridising with content features (e.g.\ gender).

\subsection{Content-Based Filtering}
\subsection*{Model Formulation}
We extend a bias-aware baseline predictor 
\[
  \hat r_{ui}^{(0)} \;=\; \mu + b_u + b_i
\]
by modeling the residual \(r_{ui} - \hat r_{ui}^{(0)}\) via content similarity.  Each profile \(i\) is represented by a standardized feature vector 
\[
  \mathbf{f}_i = \mathrm{scale}\bigl[\,
    \overline{\mathrm{res}}_i,\;
    \log(1 + \mathrm{count}_i),\;
    p_{\mathrm{female},i},\;
    p_{\mathrm{male},i},\;
    p_{\mathrm{unknown},i}
  \bigr],
\]
where \(\overline{\mathrm{res}}_i\) is the mean residual on \(i\), \(\mathrm{count}_i\) its rating count, and \(p\) the gender proportions of raters.  For user \(u\) and target \(i\), we compute
\[
  \hat r_{ui}
  = \hat r_{ui}^{(0)}
  + \frac{\sum_{j\in\mathcal N_k(i;u)} 
               \cos(\mathbf f_i,\mathbf f_j)\,(r_{uj} - \hat r_{uj}^{(0)})}
         {\sum_{j\in\mathcal N_k(i;u)} \cos(\mathbf f_i,\mathbf f_j)},
\]
where \(\mathcal N_k(i;u)\) are the \(k\) most similar profiles to \(i\) that \(u\) has rated.

\subsection*{Hyperparameter Selection}
We performed a grid-search over \(k\in\{5,10,20,50,100\}\) on a held-out validation set.  As shown in Table~\ref{tab:cbf}, \(k=50\) delivered the lowest validation MAE of 1.4898.

\subsection*{Test‐Set Evaluation}
\begin{table}[H]
  \centering
  \caption{Test MAE of Content-Based vs.\ Baseline}
  \label{tab:cbf}
  \begin{tabular}{@{}lcc@{}}
    \toprule
    Model                            & \(k\) & Test MAE \\ 
    \midrule
    Bias-aware baseline \(\mu+b_u+b_i\)     & —     & 1.5936   \\
    Bias-aware content-based (residual)     & 50    & 1.4212   \\
    \bottomrule
  \end{tabular}
\end{table}

Incorporating profile content (residual averages, popularity and gender splits) consistently improves predictions over the purely bias-based baseline, reducing test MAE by \(\approx\!0.17\).  This demonstrates that content features capture meaningful variations in user preferences.  Future work could refine feature engineering (e.g.\ by adding temporal signals or demographic covariates) or combine this approach with collaborative methods to further boost accuracy.

\section{Conclusion}
In this assignment, we evaluated three approaches to the dating profile recommendation problem using a large-scale user-rating dataset. Our exploratory analysis revealed valuable insights about user behavior and profile popularity, and preprocessing ensured the dataset was free from duplicates and efficiently structured.

The naive models provided a strong baseline, particularly the item-mean predictor, which surprisingly performed on par with more complex techniques. The collaborative filtering model, though conceptually richer, offered no significant advantage, likely due to sparse user histories and the dominance of popular profiles. The content-based model, on the other hand, showed the most promise—successfully integrating profile-level metadata such as popularity and gender distribution to outperform the baseline.

These results indicate that profile features are meaningful signals of user preference, and further improvements could be achieved by incorporating more nuanced content features or hybridizing with collaborative approaches. This project not only demonstrates the application of core data analytics techniques but also underscores the challenges and opportunities in designing recommender systems with limited contextual information.

\end{document}