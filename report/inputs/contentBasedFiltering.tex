\subsection*{Model Formulation}
We extend a bias-aware baseline predictor 
\[
  \hat r_{ui}^{(0)} \;=\; \mu + b_u + b_i
\]
by modeling the residual \(r_{ui} - \hat r_{ui}^{(0)}\) via content similarity.  Each profile \(i\) is represented by a standardized feature vector 
\[
  \mathbf{f}_i = \mathrm{scale}\bigl[\,
    \overline{\mathrm{res}}_i,\;
    \log(1 + \mathrm{count}_i),\;
    p_{\mathrm{female},i},\;
    p_{\mathrm{male},i},\;
    p_{\mathrm{unknown},i}
  \bigr],
\]
where \(\overline{\mathrm{res}}_i\) is the mean residual on \(i\), \(\mathrm{count}_i\) its rating count, and \(p\) the gender proportions of raters.  For user \(u\) and target \(i\), we compute
\[
  \hat r_{ui}
  = \hat r_{ui}^{(0)}
  + \frac{\sum_{j\in\mathcal N_k(i;u)} 
               \cos(\mathbf f_i,\mathbf f_j)\,(r_{uj} - \hat r_{uj}^{(0)})}
         {\sum_{j\in\mathcal N_k(i;u)} \cos(\mathbf f_i,\mathbf f_j)},
\]
where \(\mathcal N_k(i;u)\) are the \(k\) most similar profiles to \(i\) that \(u\) has rated.

\subsection*{Hyperparameter Selection}
We performed a grid-search over \(k\in\{5,10,20,50,100\}\) on a held-out validation set.  As shown in Table~\ref{tab:cbf}, \(k=50\) delivered the lowest validation MAE of 1.4898.

\subsection*{Test‐Set Evaluation}
\begin{table}[H]
  \centering
  \caption{Test MAE of Content-Based vs.\ Baseline}
  \label{tab:cbf}
  \begin{tabular}{@{}lcc@{}}
    \toprule
    Model                            & \(k\) & Test MAE \\ 
    \midrule
    Bias-aware baseline \(\mu+b_u+b_i\)     & —     & 1.5936   \\
    Bias-aware content-based (residual)     & 50    & 1.4212   \\
    \bottomrule
  \end{tabular}
\end{table}

Incorporating profile content (residual averages, popularity and gender splits) consistently improves predictions over the purely bias-based baseline, reducing test MAE by \(\approx\!0.17\).  This demonstrates that content features capture meaningful variations in user preferences.  Future work could refine feature engineering (e.g.\ by adding temporal signals or demographic covariates) or combine this approach with collaborative methods to further boost accuracy.