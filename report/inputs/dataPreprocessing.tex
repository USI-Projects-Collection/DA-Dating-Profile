% ---------- inputs/dataPreprocessing.tex ----------
\subsection*{Raw files and initial footprint}

The original dataset consists of three files:

\begin{itemize}
  \item \texttt{ratings.dat} – \num{3220037} user--item interactions
        (\textit{userID}, \textit{profileID}, \textit{rating})\@.
  \item \texttt{ratings\_Test.dat} – the held-out test split
        (\num{276053} rows, same schema).
  \item \texttt{gender.dat} – \num{220970} user--gender pairs.
\end{itemize}

Loaded with Pandas’ default \texttt{int64} dtypes the two training files occupy
$\sim\!90$\,MB of memory.

\subsection*{Dtype optimisation}

\vspace{2pt}
\begin{tabular}{@{}lcc@{}}
\toprule
\textbf{Column} & \textbf{Original dtype} & \textbf{Final dtype} \\ \midrule
\textit{userID}, \textit{profileID} & \texttt{int64} & \texttt{int64}\textsuperscript{*} \\
\textit{rating}                     & \texttt{int64} & \texttt{float32} \\
\textit{gender}                     & \texttt{int64} & \texttt{category} \\ \bottomrule
\end{tabular}

\smallskip
\noindent\textsuperscript{*}\,Required by \texttt{torch.nn.Embedding}.
Casting the other two columns shrinks the in-RAM size of
\texttt{ratings.dat} to $\sim\!61$\,MB and \texttt{gender.dat} to
$\sim\!2$\,MB (a 74\,\% reduction overall).

\subsection*{Duplicate removal}

A scan for exact duplicates uncovered \num{47} repeated
\textit{(user,\,profile)} pairs in the training split; these rows were
dropped, leaving \num{3219990} unique ratings.  
No missing values were present in any file.

\subsection*{Persisting the processed data}

The cleaned frames are serialised to \texttt{.pkl} with
\texttt{DataFrame.to\_pickle()}, bypassing expensive CSV parsing in every
notebook run.
% ---------- end inputs/dataPreprocessing.tex ----------
